\documentclass{exam}

% ### Packages ####
\usepackage[utf8]{inputenc}

\newcommand{\bfemph}[1]{\textbf{#1}}
\renewcommand{\emph}[1]{\bfemph{#1}}

\begin{document}

% title
\normalsize
\textsf{How do I?}
\hfill
Variables, IO, Arrays

\noindent\makebox[\linewidth]{\rule{\paperwidth}{0.4pt}}

\bigskip
\huge
\noindent
\textsf{How do I?}

% Reset font size
\large

\tableofcontents

\bigskip


\section{Basics}
\subsection{How do I: First program}
First create a file called \emph{Main.cs} and make sure to have the following in the file
\begin{verbatim}
    class Main
    {
        static void Main(string[] args) 
        {

        }
    }
\end{verbatim}
If you wish to give a different name to the file, then make sure to put that name instead of \emph{Main}
after the \emph{class} keyword.
\bigskip

\subsection{How do I: Use a variable}
Using variables is one of the main concepts of programming. Variables are used to store
data. There are different types of variables used for storing different kinds of data.

\begin{verbatim}
    int a = 5; // Integer
    double b = 3.2; // Decimal
    char ch = 'r'; // A single character, single quotes '
    string str = "Hello World"; // Array of characters - string, double quotes "

    int[] arr = new int[5]; // Defines an array of ints, of size 5
    int[] arr2 = new int[a]; // We can also use prev defined variable for size 
\end{verbatim}

\section{IO Operations}
Standard input and output (STD in, STD out) are the ways to communicate with the program
through the console window.

\subsection{Printing to STD out}

\subsection{Reading from STD in}

\vspace{5mm}

\large

\end{document}